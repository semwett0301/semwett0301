%%%%%%%%%%%%%%%%%
% Это шаблон резюме в две колонки на основе moderncv.cls
% (v1.0, 16 мая 2024), автор Genki Ogaki (rockstarogk@gmail.com). Компилируется с XeLaTeX.
%
%% Этот шаблон может распространяться и/или изменяться в соответствии с
%% условиями публичной лицензии проекта LaTeX (LaTeX Project Public License) версии 1.3
%% данной лицензии или (по вашему выбору) любой более поздней версии.
%% Последняя версия этой лицензии находится на
%%    http://www.latex-project.org/lppl.txt
%% и версия 1.3 или более поздняя входит во все дистрибутивы LaTeX
%% версии 2003/12/01 или более поздней.
%%%%%%%%%%%%%%%%

% environment
\documentclass[11pt,a4paper]{moderncv}
\moderncvtheme[blue]{banking}
\nopagenumbers{}

\usepackage[T1]{fontenc}
\usepackage{inputenc}
\usepackage[a4paper, top=1cm, bottom=0.5cm, left=0.8cm, right=0.8cm]{geometry}
\usepackage{tabularx}
\usepackage{fontawesome5}
\usepackage{paracol}
\usepackage{qrcode}
\usepackage{enumitem}  
\usepackage{multicol}
\usepackage{titlesec}

\usepackage{fontspec}
\setmainfont{Lato}

\definecolor{brandblue}{HTML}{0A66C2}
\definecolor{darkblue}{HTML}{0B1D51}

% Основной текст = 10pt (чуть плотнее межстрочный)
\renewcommand\normalsize{\fontsize{10}{12}\selectfont}
\renewcommand\small{\fontsize{9}{10}\selectfont}

% Подзаголовки = 11pt
\titleformat{\subsection}{\fontsize{11}{13}\bfseries}{}{0pt}{}

% Заголовки секций = 12pt
\titleformat{\section}
  {\vspace{2mm}\fontsize{14}{14}\color{brandblue}\bfseries\scshape}
  {}{0pt}{}
  [\titlerule\vspace{1mm}]
  
\titlespacing*{\section}{0pt}{0.8ex}{0.6ex}
\titlespacing*{\subsection}{0pt}{0.6ex}{0.4ex}

% macro
\renewcommand*{\labelitemi}{-}

\makeatletter
\renewcommand\footnotesize{\@setfontsize\footnotesize{10}{12}}
\makeatother

\newcolumntype{L}{>{\raggedright\arraybackslash}X}
\newcolumntype{C}{>{\centering\arraybackslash}X}
\newcolumntype{R}{>{\raggedleft\arraybackslash}X}

\newcommand{\educationentry}[8][]{
  \noindent
  \textbf{#4} (#7) \hfill \textbf{#2} \\[0.1em]
  \textit{#3} \hfill \textit{#6} \\[0.3em]
  \textbf{Специализация:} #5 \\[0.5em]
  \textbf{Курсы:}
  \vspace{0.5em}
  {\footnotesize
    \raggedcolumns
      \small
      \begin{itemize}[leftmargin=1.2em, itemsep=0.1em, topsep=0em, parsep=0em]
        #8
      \end{itemize}
  } 
}

% preamble
\familyname{Семён Мокров}
\address{Санкт-Петербург, Российская Федерация}{}
\title{Fullstack-разработчик}

% document
\begin{document}
\maketitle
\vspace{-7mm}

\section{КОНТАКТЫ}
\begin{tabular*}{\textwidth}{@{\extracolsep{\fill}} l l l @{}}
  \emailsymbol\enspace \emaillink{mokrovsimon@gmail.com} &
  \mobilesymbol\enspace +31 (06) 579-875-47 &
  \faLinkedin\enspace \href{https://www.linkedin.com/in/sem-wett/}{linkedin.com/in/sem-wett} \\[4pt]

  \emailsymbol\enspace \emaillink{simon.mokrov@hva.nl} &
  \faWhatsapp\enspace \href{https://wa.me/310657987547}{+31 (06) 579 875 47} &
  \faTelegram\enspace \href{https://t.me/sem_wett}{sem\_wett} \\[4pt]

  \faGithub\enspace \href{https://github.com/semwett0301}{github.com/semwett0301} &
  \faGitlab\enspace \href{https://gitlab.com/semen.mokrov}{gitlab.com/semen.mokrov} &
  \faLanguage\enspace Английский \,|\, Русский \,|\, Нидерландский (A1) \\
\end{tabular*}

\section{ОБО МНЕ}
\textbf{Fullstack-разработчик} с \textbf{3+ годами опыта} разработки масштабируемых, высокопроизводительных приложений во фронтенде и бэкенде. Обладает навыками JavaScript/TypeScript, Java (Spring Boot), React, Redux, NestJS, PostgreSQL и Docker, имеет успешный опыт работы в Agile-командах. \\ [0.3em]
\textbf{Профессиональный опыт} охватывает ритейл, EdTech и банковское ПО, с подтверждённым влиянием на производительность, внедрение новых функций и масштабируемость систем. Магистр программы Digital Driven Business (AUAS, Амстердам) со специализацией в data-driven разработке, машинном обучении и статистике. \\ [0.3em]
Увлечён созданием надёжных и масштабируемых систем, открыт для возможностей как во фронтенде, так и в бэкенде с использованием full-stack экспертизы.

\section{ОПЫТ (> 3 ЛЕТ)}

\begin{enumerate}[leftmargin=1em, itemsep=0.01em, labelsep=0.2em, topsep=0em]

  \item \textbf{\color{darkblue} Фронтенд-разработчик} \hfill \textit{март 2024 -- май 2025} \\ [0.5mm]
  \small{\textit{\textbf{X5 Tech} --- крупнейший ритейлер в России}} \\ [1mm]
  \small{Разрабатывал и поддерживал сервисы подписки («Пакет», «Абонемент») в корпоративной кросс-функциональной среде.} \\ [1mm]
  {\small \textbf{Стек:} React, TanStack Query, pnpm, Axios, Chart.js, Python \& FastAPI (чтение)}
  {\small
    \vspace{-0.5mm}
    \begin{itemize}[leftmargin=1.2em, itemsep=0.1em, topsep=0em, partopsep=0pt]
      \item Увеличил удержание клиентов на 12\% благодаря улучшениям фронтенда.
      \item Сократил время вывода релизов на рынок на 15\% за 6 месяцев.
      \item Ускорил внедрение функций на 10\% благодаря Scrum-практикам.
      \item Снизил количество ошибок в продакшене на 18\% после рефакторинга кода.
    \end{itemize}
  }

  \item \normalsize{\textbf{\color{darkblue} Бэкенд-разработчик} \hfill \textit{июль 2024 -- фев 2025}} \\ [0.5mm]
  \small{\textit{\textbf{Университет ИТМО} --- банковское ПО для Газпромбанка}} \\ [1mm]
  \small{Усовершенствовал внутренние системы для банковских гарантий, оптимизировав бэкенд-процессы и обработку корпоративных данных.} \\ [1mm]
  {\small \textbf{Стек:} Kotlin, Spring Boot, Gradle, PostgreSQL, Kafka, SOAP, Docker} 
  {\small
    \vspace{-0.5mm}
    \begin{itemize}[leftmargin=1.2em, itemsep=0.1em, topsep=0pt, partopsep=0pt]
      \item Сократил среднюю задержку откликов на 12\% с помощью рефакторинга эндпойнтов.
      \item Ускорил выполнение PostgreSQL-запросов до 20\%.
      \item Сократил время формирования отчёта с 90 до 60 секунд.
      \item Снизил нагрузку на сервер на 20\% после внедрения Kafka.
    \end{itemize}
  }

  \item \normalsize{\textbf{\color{darkblue} Фулстек-разработчик} \hfill \textit{ноя 2023 -- апр 2024}} \\ [0.5mm]
  \small{\textit{\textbf{Tune IT} --- EdTech-решения}} \\ [1mm]
  \small{Разрабатывал и поддерживал образовательную платформу Skillfactory и систему Polytech для абитуриентов, сосредоточившись на масштабируемости, удобстве и производительности.} \\ [1mm]
  {\small \textbf{Стек:} React, TypeScript, RTK Query, SCSS, Styled Components, Spring Cloud, Kotlin} 
  {\small
    \vspace{-0.5mm}
    \begin{itemize}[leftmargin=1.2em, itemsep=0.1em, topsep=0pt, partopsep=0pt]
      \item Решил 30+ проблем продукта, улучшив стабильность.
      \item Сократил цикл «дизайн → разработка» на 25\% благодаря кастомным UI-компонентам.
      \item Снизил задержку сервисов на 13\% через модульность на Spring Cloud.
    \end{itemize}
  }

  \item \normalsize{\textbf{\color{darkblue} Фулстек-разработчик} \hfill \textit{мар 2023 -- ноя 2023}} \\ [0.5mm]
  \small{\textit{\textbf{MagicGophers} --- социальные приложения для VK}} \\ [1mm]
  \small{Создавал VK мини-приложения для сторонних заказчиков, совмещая фронтенд и бэкенд.} \\ [1mm]
  {\small \textbf{Стек:} React, VK UI, VK Bridge, VK Router, VK API, NestJS, PostgreSQL, Redis} 
  {\small
    \vspace{-0.5mm}
    \begin{itemize}[leftmargin=1.2em, itemsep=0.1em, topsep=0pt, partopsep=0pt]
      \item Создал приложения, которыми воспользовались 10\,000+ пользователей VK.
      \item Повысил конверсию на 20\% благодаря улучшениям UI/UX.
      \item Сократил количество баг-репортов на 30\% за 6 месяцев.
      \item Оптимизировал запросы к БД, уменьшив задержку на 25\%.
    \end{itemize}
  }

  \item \normalsize{\textbf{\color{darkblue} Фронтенд-разработчик} \hfill \textit{июль 2022 -- мар 2023}} \\ [0.5mm]
  \small{\textit{\textbf{Университет ИТМО} --- EdTech (дистанционный прокторинг)}} \\ [1mm]
  \small{Переработал фронтенд для ITMOproctor, улучшив масштабируемость и UX.} \\ [1mm]
  {\small \textbf{Стек:} React, Redux, Vue.js, JavaScript, WebRTC, WebSockets, Axios} 
  {\small
    \vspace{-0.5mm}
    \begin{itemize}[leftmargin=1.2em, itemsep=0.1em, topsep=0pt, partopsep=0pt]
      \item Сократил time-to-market новых функций на 20\% благодаря модульной архитектуре.
      \item Уменьшил время загрузки на 10\% через оптимизацию маршрутизации, состояния и сокетов.
      \item Обеспечил защищённые трансляции для 5\,000+ экзаменов в реальном времени с использованием WebRTC.
    \end{itemize}
  }

\end{enumerate}

\section{ДОПОЛНИТЕЛЬНЫЙ ОПЫТ}
\begin{enumerate}[leftmargin=1em, itemsep=0.01em, labelsep=0.2em]
  \item \normalsize{\textbf{\color{darkblue} Ассистент преподавателя, Университет ИТМО} \hfill \textit{сен 2022 -- фев 2023}} \\
  {\small \textbf{Стек:} Java, Spring Boot, PHP, JavaScript, React, Vue.js, Angular, HTML, CSS \\
  Помогал студентам первого курса в программировании и веб-разработке. Проводил занятия по Java и курировал full-stack проекты.}

  \item \normalsize{\textbf{\color{darkblue} CMS на FastAPI}} \hfill \textit{2022} \\
   {\small \textbf{Стек:} Python, FastAPI, PostgreSQL, HTML, CSS, JavaScript \\
  Разработал CMS в командном проекте с использованием FastAPI и PostgreSQL. Реализовал масштабируемое решение с аутентификацией, моделированием данных и кастомной админкой.}
\end{enumerate}

\vspace{-1em}

\section{НАВЫКИ}
{\renewcommand{\arraystretch}{1.05}
\setlength{\tabcolsep}{4pt} % Увеличиваем расстояние между колонками
\begin{tabularx}{\linewidth}{@{}>{\bfseries}l >{}X@{}}
Languages & JavaScript, TypeScript, Java, Kotlin, Python, C \\ [0.25em]
Frontend  & React, Redux (TanStack/RTK), Vue.js, Next.js, Electron, React Native, Expo, Styled Components, CSS3, HTML5 \\ [0.25em]
Backend   & Spring Boot (Data, Security, MVC), Java EE, NestJS, FastAPI, gRPC \\ [0.25em]
Messaging & Kafka, RabbitMQ, Amazon SQS, WebSockets, SSE, GraphQL \\ [0.25em]
Databases & PostgreSQL, Oracle, MongoDB, Redis\\ [0.25em]
DevOps    & Docker, Git, CI/CD, AWS, Nginx, webpack, vite, monorepo \\ [0.25em]
ML & scikit-learn, pandas, PyTorch, XGBoost, NumPy, MCP
\end{tabularx}
}
\section{ОБРАЗОВАНИЕ}
\vspace{0.5em}
\newlength{\EduGap}
\setlength{\EduGap}{4mm}

\noindent
\begin{minipage}[t]{\dimexpr0.5\linewidth-0.5\EduGap\relax}
  \educationentry[]
    {AUAS (HvA)}
    {Амстердам, Нидерланды}
    {Магистр}
    {Digital Driven Business}
    {2024 -- 2025}
    {MSc}
    {
      \item Статистическое тестирование
      \item Машинное обучение
      \item Кластеризация
      \item Глубокое обучение
      \item Scrapy \& Selenium
      \item Рекомендательные системы
      \item Модели множественной линейной регрессии
      \item Исследовательские навыки
    }

  \vspace{0.2em}\dotfill\vspace{0.7em}

  \educationentry[]
    {Ozyegin University}
    {Стамбул, Турция}
    {Бакалавр}
    {Computer Science (exchange)}
    {2023 -- 2023}
    {BSc -- Exchange semester}
    {
      \item Testing and analyzis
      \item Parallel Computing
      \item Advanced Databases
      \item Computer Networks
    }
\end{minipage}
\hspace{\EduGap}%
\begin{minipage}[t]{\dimexpr0.5\linewidth-0.5\EduGap\relax}
  \educationentry[]
    {Университет ИТМО}
    {Санкт-Петербург, Россия}
    {Бакалавр}
    {Программная инженерия}
    {2020 -- 2024}
    {BSc}
    {
       \item Высшая и вычислительная математика
       \item Теория вероятностей и теория очередей
       \item Веб и низкоуровневое программирование
       \item Алгоритмы и структуры данных
       \item Операционные системы
       \item Базы данных
    }
\end{minipage}
\end{document}